\documentclass[11pt,a4paper]{scrartcl}
\usepackage[utf8]{inputenc}
\usepackage[T1]{fontenc}
\usepackage{lmodern}
\usepackage[australian,american]{babel}

%\usepackage[font=footnotesize]{subcaption}

%\usepackage{amsmath}
%\usepackage{amsfonts}
%\usepackage{amssymb}
\usepackage[headsepline,footsepline]{scrlayer-scrpage}
\usepackage{minted}
%\usepackage{tabu}
%\usepackage{longtable}
\usepackage{multirow}
\usepackage{setspace}
%\usepackage{graphicx}
%\usepackage{pgfplots}
\usepackage{url}
\usepackage{titling}
%\usepackage[backend=biber,style=ieee,bibencoding=utf8,sorting=none]{biblatex}
\usepackage{csquotes}
\usepackage{pdfpages}

%\usepackage{epstopdf}
\usepackage{graphicx}
\usepackage{subfig}
\usepackage{todonotes}
\usepackage{pdflscape}
\usepackage[hidelinks]{hyperref}

%\usepackage{enumitem}

%\pgfplotsset{width=12cm,height=6cm,compat=1.11}

% Constants
\def\mytitle{Gossip-based dissemination, Peer Sampling Service}
\def\myauthor{Sébastien Vaucher}

\pagestyle{scrheadings}

\ihead{\mytitle}
\chead{}
\ohead{\myauthor}
\cfoot{}
\ifoot{Large-Scale Distributed Systems}
\ofoot{\thepage}

\posttitle{\end{center}\begin{center}\LARGE Large-Scale Distributed Systems\end{center}}

\author{\myauthor\\ \href{mailto:sebastien.vaucher@unine.ch}{sebastien.vaucher@unine.ch}}
\title{\huge \textbf{Gossip-based dissemination,\\ Peer Sampling Service}}

\hypersetup{
	pdftitle=\mytitle,
	pdfauthor=\myauthor
}

\begin{document}

\nocite{*}

\begin{titlingpage}

\begin{otherlanguage}{australian}
\maketitle
\end{otherlanguage}

\setcounter{tocdepth}{2}
\tableofcontents
%\listoffigures

\begin{figure}[b]
\centering
\subfloat{\includegraphics[height=1.5cm]{jmcs.png}}
\qquad
\subfloat{\includegraphics[height=1.8cm]{unine.pdf}}
\end{figure}

\end{titlingpage}

\pagebreak

\section{Introduction}

This report presents the results obtained in the first assignment of the Large-Scale Distributed Systems course taught at the University of Neuchâtel. The goal of the assignment is the implement two gossip-based dissemination algorithms, namely anti-entropy and rumor mongering. A second part of the assignment consists in implementing a peer-sampling service. The implementations are programmed in the Lua programming language, and use the Splay framework.

Apart from this report, a number of files are supplied:

\begin{description}
\item[gossip.lua]\hfill\\ Contains the entry-point of the program and the implementations of both gossip algorithms.
\item[pss.lua]\hfill\\ Contains the implementation of the peer-sampling service.
\item[gossip.sh]\hfill\\ Bash script to launch the program on a local machine.
\item[parse\_log.pl]\hfill\\ Perl script to parse logs produced by the program and generate files readable by the Gnuplot utility.
\item[gnuplot\_*.gp]\hfill\\ Gnuplot scripts generating the graphs found in this report.
\item[*.txt]\hfill\\ Raw logs generated by the subsequent program launches.
\item[generate\_graphs(.sh|.bat)] Script (\textsf{.sh} for POSIX-compliant systems, \textsf{.bat} for Windows) to generate the plots found in this report from the raw logs. Note that only the \textsf{.sh} version is able to generate the plots relative to the peer-sampling service.
\end{description}


\section{Gossip-based dissemination}

As part of the assignment, the anti-entropy and rumor mongering algorithms were implemented. The source code of both is contained in the file \textsf{gossip.lua}. The goal of this particular implementation is to disseminate an infection to 40 individual nodes.

\subsection{Plots}

In \autoref{fig:223-1} and \autoref{fig:223-2}, we compare the speed at which the infection reaches the nodes. \autoref{fig:223-1} shows the difference when we vary the $HTL$ parameter of the rumor mongering algorithm. This \textit{Hops To Live} value specifies how many nodes a message traverses before it stops being transmitted further. \autoref{fig:223-2}, in the other hand, shows the effect of the $F$ parameter. It states to how many random peers a given node has to propagate an incoming message. Both figures also display the performance of the anti-entropy algorithm.

Figures \ref{fig:223-1-dup}, \ref{fig:223-2-dup} and \ref{fig:223-3} display the number of duplicates metric. \autoref{fig:223-1-dup} and \ref{fig:223-2-dup} show the same execution as \autoref{fig:223-1} and \ref{fig:223-2}, respectively. The number of duplicates metric is applied to the rumor mongering algorithm. It is incremented by 1 every time a node receives a message that it also knows about. \autoref{fig:223-3} shows the number of duplicate messages needed to reach a certain number of nodes. Note that the x axis is a logarithmic scale.

\subsection{Analysis}

\begin{figure}
	\centering
	\includegraphics[width=0.9\linewidth]{223_1.pdf}
	\caption{Variation of the $HTL$ parameter, percentage of infected nodes over time}
	\label{fig:223-1}
\end{figure}
\begin{figure}
	\centering
	\includegraphics[width=0.9\linewidth]{223_1_dup.pdf}
	\caption{Variation of the $HTL$ parameter, duplicate messages over time}
	\label{fig:223-1-dup}
\end{figure}
\begin{figure}
	\centering
	\includegraphics[width=0.9\linewidth]{223_2.pdf}
	\caption{Variation of the $F$ parameter, percentage of infected nodes over time}
	\label{fig:223-2}
\end{figure}
\begin{figure}
	\centering
	\includegraphics[width=0.9\linewidth]{223_2_dup.pdf}
	\caption{Variation of the $F$ parameter, duplicate messages over time}
	\label{fig:223-2-dup}
\end{figure}
\begin{figure}
	\centering
	\includegraphics[width=0.9\linewidth]{223_3.pdf}
	\caption{Number of duplicate messages needed to achieve a given number of infections}
	\label{fig:223-3}
\end{figure}

\section{Peer-sampling service}



\begin{figure}
	\centering
	\includegraphics[width=0.9\linewidth]{clustering.pdf}
	\caption{Clustering of the nodes}
	\label{fig:clustering}
\end{figure}
\begin{figure}
	\centering
	\includegraphics[width=0.9\linewidth]{indegrees.pdf}
	\caption{Cumulative in-degree of the nodes}
	\label{fig:indegrees}
\end{figure}

\begin{figure}
	\centering
	\includegraphics[width=0.9\linewidth]{32.pdf}
	\caption{Dissemination using the peer-sampling service}
	\label{fig:32}
\end{figure}
\begin{figure}
	\centering
	\includegraphics[width=0.9\linewidth]{32_dupnodes.pdf}
	\caption{Duplicated messages while disseminating using the peer-sampling service}
	\label{fig:32-dup}
\end{figure}

\end{document}