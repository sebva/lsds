\documentclass[11pt,a4paper]{scrartcl}
\usepackage[utf8]{inputenc}
\usepackage[T1]{fontenc}
\usepackage{lmodern}
\usepackage[australian,american]{babel}

%\usepackage[font=footnotesize]{subcaption}

%\usepackage{amsmath}
%\usepackage{amsfonts}
%\usepackage{amssymb}
\usepackage[headsepline,footsepline,automark]{scrlayer-scrpage}
\usepackage{minted}
%\usepackage{tabu}
%\usepackage{longtable}
\usepackage{multirow}
\usepackage{setspace}
%\usepackage{graphicx}
%\usepackage{pgfplots}
\usepackage{url}
\usepackage{titling}
%\usepackage[backend=biber,style=ieee,bibencoding=utf8,sorting=none]{biblatex}
\usepackage{csquotes}
\usepackage{pdfpages}
\usepackage[nomarkers,figuresonly]{endfloat}

%\usepackage{epstopdf}
\usepackage{graphicx}
\usepackage{subfig}
\usepackage{todonotes}
\usepackage{pdflscape}
\usepackage[hidelinks]{hyperref}

%\usepackage{enumitem}

%\pgfplotsset{width=12cm,height=6cm,compat=1.11}

% Constants
\def\mytitle{Distributed Hash Tables}
\def\myauthor{Sébastien Vaucher}

\pagestyle{scrheadings}

\ihead{\headmark}
\chead{}
\ohead{\myauthor}
\cfoot{}
\ifoot{Large-Scale Distributed Systems, Assignment 2}
\ofoot{\thepage}

\posttitle{\end{center}\begin{center}\LARGE Large-Scale Distributed Systems\end{center}}

\author{\myauthor\\ \href{mailto:sebastien.vaucher@unine.ch}{sebastien.vaucher@unine.ch}}
\title{\huge \textbf{\mytitle}}

\hypersetup{
	pdftitle=\mytitle,
	pdfauthor=\myauthor
}

\renewcommand{\efloatseparator}{\mbox{}}

\begin{document}

\nocite{*}

\begin{titlingpage}

\begin{otherlanguage}{australian}
\maketitle
\end{otherlanguage}

%\setcounter{tocdepth}{1}
\tableofcontents
%\listoffigures

\begin{table}[b]
\centering
\subfloat{\includegraphics[height=1.3cm]{jmcs.png}}
\qquad\qquad
\subfloat{\includegraphics[height=1.5cm]{unine.pdf}}
\end{table}

\end{titlingpage}

\pagebreak

\section{Introduction}

This report presents the results obtained in the second assignment of the Large-Scale Distributed Systems course taught at the University of Neuchâtel. The goal of the assignment is the implement a distributed hash table using the Chord algorithm. The implementation is done in the Lua programming language, using the Splay framework.

Apart from this report, a number of files are supplied:

\begin{description}
\item[dht.lua]\hfill\\ Contains the entry-point of the program and the implementations of the DHT.
\end{description}

All the data presented in this report is the result of executions on the Splay cluster of the university.

For a better reading comfort, all figures are disposed at the end of the document.



%\begin{figure}
%	\centering
%	\includegraphics[width=0.93\linewidth]{223_1.pdf}
%	\caption{Variation of the $HTL$ parameter, percentage of infected nodes over time}
%	\label{fig:223-1}
%\end{figure}

\end{document}